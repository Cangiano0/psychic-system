
\documentclass[a4paper, 12pt]{article}
\usepackage[portuges]{babel}
\usepackage[utf8x]{inputenc}
\usepackage[T1]{fontenc}
\usepackage[a4paper,top=3cm,bottom=2cm,left=3cm,right=3cm,marginparwidth=1.75cm]{geometry} 
\usepackage{float}
\usepackage{graphicx}
\usepackage{caption}
\usepackage{wrapfig}

\usepackage[colorlinks=False]{hyperref}

\usepackage{amsmath, amsfonts, amssymb} 
\usepackage{color} 
\usepackage{enumitem}
\usepackage{minted} 


\begin{document}

\section*{Questão 1 - Genética de Populações}
\large {Suponha uma população em que a ocorrência do gene recessivo Xd, relacionado à expressão do dautonismo, tenha frequência igual a 0,001 e a do gene Y tenha 0,5. Pede-se:}
\newline
\newline
\textbf{A.} Escreva os possiveis genótipos e os respectivos fenótipos, identificando o sexo e a expressão ou não de dautonismo.
\newline
\newline
\textbf{B.} Partindo do item anterior, forneça as frequências de cada fenótipo.
\newline
\newline
\textbf{C.}  Calcule a probabilidade, em um conjunto de 10 pessoas escolhidas arbitrariamente nessa população, de termos:
\begin{itemize}
\item{5 pessoas dautônicas.}
\item{Pelo menos uma pessoa dautônica.}
\item{Apenas uma pessoa dautônica.}
\end{itemize}
\vspace{4mm}
\textbf{D.} Generalize os resultados acima para um grupo de N pessoas e um numero K de pessoas dautônicas, para cada um dos três itens.



















\end{document}
